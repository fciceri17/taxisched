\section{Risks associated with the project}
  \subsection{Project Risk}
    Here is a brief description of risks we may run into due to poor design.
    \begin{description}
      \item[Delays over expected deadlines:] the project may be harder than we estimated. This may cause a delayed release or a release of a incomplete version.
      \item[Lack of experience:] the team may lack in programming experience this may cause a possible release delay.
      \item[Structural changes:] during development we may find our project guide line hard to be implemented and we may have to restructure.
      \item[Lack of communication:] to develop this project in time the team needs to communicate; this can be hard because team members may work remotely. This can lead to misunderstandings.
          To minimize this we have to make a RASD and DD to guide the development.
    \end{description}

    \begin{table}[h]
    \centering
        \begin{tabular}{| l | l | l |}
            \hline
            \textbf{Risk} & \textbf{Probability} & \textbf{Effects}  \\
            \hline
            Delays Over Expected Deadlines & High & Moderate \\
            \hline
            Lack of Experience & High & Moderate \\
            \hline
            Structural Changes & Low & Moderate \\
            \hline
            Lack of Communication & High & Moderate \\
            \hline
        \end{tabular}
        \caption{Evaluation of Project risks.}
    \end{table}

  \subsection{Technical Risk}
  Here is a brief description of risks we may have due to poor implementation.

    \begin{description}
      \item[Server downtime:] we may underestimate the system load or find
          software bugs. This can cause server downtime. We can work around
          the software bugs with load testing or by using third party cloud-based servers with redundancy.
      \item[Market reach:] the end user may not want to use our taxi service. 
      \item[Data security:] we can have data lost or leaked by hardware failure,
          software bugs or third party attacks. We can avoid
          this by using security standards for our software and by testing it.
      \item[Scalability:] the system could have problem with a larger number of users. We can use a third party cloud server
          to host our system and aid the growth process.
      \item[Bad code:] the code in large project may become hard to read/understand. We can avoid this by
          write a Design Document to guide our development.
    \end{description}

    \begin{table}[h]
      \centering
          \begin{tabular}{| l | l | l |}
              \hline
              \textbf{Risk} & \textbf{Probability}  & \textbf{Effects}  \\
              \hline
              Server Downtime & Moderate & High\\
              \hline
              Market Reach & Moderate & Moderate\\
              \hline
              Data Security & Moderate & High\\
              \hline
              Scalability & Low & Moderate          \\
              \hline
              Bad Code & Moderate & Moderate\\
              \hline
          \end{tabular}
        \caption{Evaluation of technical risks.}
    \end{table}
	
  \subsection{Economical Risk}
    \begin{description}
      \item[Bankruptcy] the city may withdraw their offer. We can't avoid this.
      \item[Wrong cost evaluation] the cost may exceed our budget. To prevent this we may run a very in-depth analysis to prevent this risk.
    \end{description}

    \begin{table}[h]
\centering
    \begin{tabular}{| l | l | l |}
        \hline
        \textbf{Risk}                   & \textbf{Probability}  & \textbf{Effects}  \\
        \hline
        Bankruptcy & Low & High \\
        \hline
        Wrong Cost Evaluation & Moderate & High \\
        \hline
    \end{tabular}
    \caption{Evaluation of economical risks.}
\end{table}
