\section{Function Points and COCOMO}

\subsection{Function Points}
	\subsubsection{ILF - Internal Logic Files}
		The ILFs of our system will be the following:
		\begin{itemize}
			\item Guest
			\item User
			\item Taxi Driver
			\item Ride
		\end{itemize}
	\subsubsection{ELF - External Logic Files}
		The ELFs of our system will be the following:
		\begin{itemize}
			\item External Maps
		\end{itemize}
	\subsubsection{EI - External Input}
		The EIs of our system will be the following:
		\begin{itemize}
			\item Signup
			\item Login/Logout
			\item User Information Modification
			\item Ride Booking
			\item Ride Modification/Cancellation
			\item Driver Status Modification
		\end{itemize}
	\subsubsection{EQ - External Enquiry}
		The EQs of our system will be the following:
		\begin{itemize}
			\item Ride History
			\item Ride Information
			\item User Information
			\item Driver Request Response
		\end{itemize}
	\subsubsection{EO - External Outputs}
		The EQs of our system will be the following:
		\begin{itemize}
			\item Ride History Creation
			\item Shared Ride
			\item User Notification
		\end{itemize}
		
	\subsubsection{Complexity Estimation}
		\begin{tabular}{|p{4cm}|p{9cm}|}
			\hline
			\textbf{ILF} & \textbf{Complexity}\\
			\hline
			\textbf{Guest} & Simple \\
			\hline
			\textbf{User} & Medium \\
			\hline
			\textbf{Taxi Driver} & Medium \\
			\hline
			\textbf{Ride} & High \\
			\hline
		\end{tabular}
		
		\vspace{2em}

		\noindent\begin{tabular}{|p{4cm}|p{9cm}|}
			\hline
			\textbf{ELF} & \textbf{Complexity}\\
			\hline
			\textbf{External Maps} & High \\
			\hline
		\end{tabular}
		
		\vspace{2em}
		
		\noindent\begin{tabular}{|p{4cm}|p{9cm}|}
			\hline
			\textbf{EI} & \textbf{Complexity}\\
			\hline
			\textbf{Signup} & Medium \\
			\hline
			\textbf{Login} & Simple \\
			\hline
			\textbf{Logout} & Simple \\
			\hline
			\textbf{User Information Modification} & Medium \\
			\hline
			\textbf{Ride Booking} & Medium \\
			\hline
			\textbf{Ride Modification} & Medium \\
			\hline
			\textbf{Ride Cancellation} & Simple \\
			\hline
			\textbf{Driver Status Modification} & Simple \\
			\hline
		\end{tabular}
		
		\vspace{2em}
		
		\noindent\begin{tabular}{|p{4cm}|p{9cm}|}
			\hline
			\textbf{EQ} & \textbf{Complexity}\\
			\hline
			\textbf{Ride History} & Simple \\
			\hline
			\textbf{Ride Information} & Simple \\
			\hline
			\textbf{User Information} & Simple \\
			\hline
			\textbf{Driver Request Response} & Simple \\
			\hline
		\end{tabular}
		
		\vspace{2em}
		
		\noindent\begin{tabular}{|p{4cm}|p{9cm}|}
			\hline
			\textbf{EO} & \textbf{Complexity}\\
			\hline
			\textbf{Ride History Creation} & High \\
			\hline
			\textbf{Shared Ride} & High \\
			\hline
			\textbf{User Notification} & Simple \\
			\hline
		\end{tabular}
		\\
		\\
		
	\noindent This is the table we refer to assign cost to the Function Points\\
	\\
	
	
	\begin{tabular}{|l|l|l|l|}
	\hline
	\multirow{2}*{\textbf{Function Types}} & \multicolumn{3}{|c|}{\textbf{Complexity}}\\
	\cline{2-4}
	& \textbf{Simple} & \textbf{Medium} & \textbf{High}\\
	\hline
	N.EI & 3 & 4 & 6 \\
	\hline
	N.EO & 4 & 5 & 7 \\
	\hline
	N.EQ & 3 & 4 & 6 \\
	\hline
	N.ILF & 7 & 10 & 15 \\
	\hline
	N.EIF & 5 & 7 & 10 \\
	\hline
	\end{tabular}
	
	\vspace{2em}
	
\noindent So we have:\\

\begin{tabular}{|l|l|l|l|l|}
	\hline
	\multirow{2}*{\textbf{Function Types}} & \multicolumn{3}{|c|}{\textbf{Number}} & \multirow{2}*{\textbf{Points}}\\
	\cline{2-4}
	& \textbf{Simple} & \textbf{Medium} & \textbf{High} & \\
	\hline
	EI & 4 & 4 & 0 & 28 \\
	\hline
	EO & 1 & 0 & 3 & 25\\
	\hline
	EQ & 4 & 0 & 0 & 12\\
	\hline
	ILF & 1 & 2 & 3 & 32\\
	\hline
	EIF & 0 & 0 & 1 & 10\\
	\hline
	\multicolumn{4}{|l|}{\textbf{TOTAL}} & 107\\
	\hline
	\textbf{SLOC} & \multicolumn{3}{|c|}{x53} & 5671\\
	\hline
	\end{tabular}

	\vspace{2em}
	
The multiplicator to convert UFP to SLOC for Java language is 53.
	
\newpage	
\subsection{COCOMO}

To calculate the Effort of our project, we will use COCOMO II model. \\
The formula that calculates the Effort is:\\
\begin{center}
Effort $= A \times EAF \times KSLOC^E$\\
\end{center}
Where: 
\begin{itemize}
\item EAF = $\begin{matrix} \prod_{i} C_i \end{matrix}$ with $C_i$ a single Cost Driver\\
\item E = $ 0.91 + 0.01 \times \begin{matrix} \prod_{i} SF_i \end{matrix}$ with $SF_i$ a single Scale Driver\\
\item KSLOC = Kilo Source Lines of Code\\
\item A = 2.94\\
\end{itemize}
So now we need to analize the Scale Drivers and the Cost Drivers.\\

\newpage
\subsubsection{Scale Drivers}
Scale Drivers are parameters that non-linearly influence the effort, in relation to the Lines of Code.\\
There are five types of these parameters, each one can go to \texttt{Very low} to \texttt{Extra high}:\\
\begin{itemize}
\item{\textbf{Precedentedness:}} this parameter reflects the previous experience on this type of project of the people that are working on it. \texttt{Very low} means no experience at all, \texttt{Extra high} means complete familiarity.
	In our project this parameter is \texttt{Low} because we have some experience in project desing but most of these issues are new to us.\\
\item{\textbf{Development flexibility:}} explains the level of flexibility in the development process. \texttt{Very low} means that the developer has been given rigorous requests and clear goals, \texttt{Extra high} means that were given only generic goals.
	We set it to \texttt{Nominal} because we were in a not too strict situation, where we had clear goals but we could interpret them with a certain degree of freedom.\\
\item{\textbf{Risk resolution:}} evaluates the risk assessment. We set it to \texttt{Nominal}, since our risk analysis is not too extensive.
\item{\textbf{Team Cohesion:}} it obviously identifies the cohesion of the team members. In our case we set it to \texttt{High} since we know each other quite well but it's the first time we work together in a group of three people.\\
\item{\textbf{Process Maturity:}} Reflects the process maturity of the organisation. The computation of this value depends on the CMM Maturity Questionnaire but we estimated it at \texttt{High} that corresponds to CMM Level 3.\\
\end{itemize}




	\begin{tabular}{|l|l|l|l|}
	\hline
	\textbf{Code} & \textbf{Name} & \textbf{Factor} & \textbf{Value}\\
	\hline
	PREC & Precedentedness & Low & 4.96\\
	\hline
	FLEX & Development flexibility & Nominal & 3.04\\
	\hline
	RESL & Risk resolution & Nominal & 4.24\\
	\hline
	TEAM & Team cohesion & High & 2.19\\
	\hline
	PMAT & Process maturity & High & 3.12\\
	\hline
	\textbf{Total} & \multicolumn{2}{|c|}{E = $ 0.91 + 0.01 \times \begin{matrix} \prod_{i} SF_i \end{matrix}$} & 1.09\\
	\hline	
	\end{tabular}
	
	\vspace{2em}

\subsubsection{Cost Drivers}

COCOMO II has 17 Cost Drivers that are multiplicative factors that reflect some characteristics of the developing process. 
The range of the values is the same as the Scale Drivers.
The table below shows our project's values for the Cost Drivers.\\ 
	
	\begin{tabular}{|l|l|l|l|}
	\hline
	\textbf{Code} & \textbf{Name} & \textbf{Factor} & \textbf{Value}\\
	\hline
	RELY & Required Software Reliability			& High 	&		1.10\\
	\hline
	DATA & Data base size 							& Nominal &		1.00\\
	\hline
	CPLX & Product Complexity 						& Nominal &		1.00\\
	\hline
	RUSE& Required Reusability 						& High &		1.07\\
	\hline
	DOCU & Documentation match to life-cycle needs 	& Low & 		0.91\\
	\hline
	TIME & Execution Time Constraint 				& Very High &	1.29\\
	\hline
	STOR & Main Storage Constraint 					& Nominal &		1.00\\
	\hline
	PVOL & Platform Volatility 						& Low & 		0.87\\
	\hline
	ACAP & Analyst Capability 						& Nominal & 	1.00\\
	\hline
	PCAP & Programmer Capability 					& Nominal &		1.00\\
	\hline
	APEX & Application Experience 					& Very Low &	1.22\\
	\hline
	PLEX & Platform Experience 						& Nominal &		1.00\\
	\hline
	LTEX & Language and Tool Experience 			& Low & 		1.09\\
	\hline
	PCON & Personnel Continuity 					& High & 		0.90\\
	\hline
	TOOL & Usage of Software Tools 					& Nominal &		1.00\\
	\hline
	SITE & Multisite Development 					& High &		0.93\\
	\hline
	SCED & Required Development Schedule 			& High &		1.00\\
	\hline
	\textbf{Total} & \multicolumn{2}{|c|}{EAF = $\begin{matrix} \prod_{i} C_i \end{matrix}$} & 1.34 \\
	\hline
	\end{tabular}
	
	
\subsubsection{Effort Estimate}
	Our effort estimate amounts to \textbf{26.1 person-months}.  
	
\subsubsection{Duration}
	The Duration of our project is given by this equation:\\
	\begin{center}
	
	Duration = $ 3.67 \times  ($Effort$)^{0.28 + 0.2 \times (E-0.91)} $ \\
	
	\end{center}
	
\noindent Where the variables are the same as in the Effort Equation.\\
\newline
	The estimate of the duration of the project amounts to \textbf{10.2 months}. This estimate gives us a team size of 2.56 people.
	With these numbers in mind, we can expect a 10 months time frame to be a generous estimate, due to our team size being 3.
	
	