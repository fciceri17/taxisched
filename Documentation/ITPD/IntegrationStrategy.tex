\section{Integration Strategy}
\subsection{Entry Criteria}
We assume that all the functions have been unit tested and that every component is working well. Our aim is to test interactions between each component so we need that every part of our system to not have internal problems.
\subsection{Elements to be Integrated}
The components we want to test are the ones described in our Design Document, more precisely in the Component View.\newline
This is the list of the Components with their SubComponents whose interactions need to be tested: \newline
\newpage

\noindentation\textbf{Client Components}
\begin {itemize}
\item Client Manager
\end {itemize}
\textbf{Ride Manager Components}
\begin{itemize}
\item Ride Controller
\item Queue Manager
\item Shared Ride Builder
\item Database Manager
\end {itemize}
\textbf{Data Manager Components}
\begin {itemize}
\item Data Controller
\item Authentication Manager
\item Profile Editor
\end{itemize}

\begin{figure}[h!]
  \centering
  \includegraphics[width=0.9\textwidth]{componentsTest/TestComponents}
\end{figure}

\subsection{Integration Testing Strategy}
We decided to use the bottom-up approach. We thought this was the best strategy in order to better test the connection between each level of our project.
Before testing the top level we need to be sure that lower levels work in order to understand where hypothetical problems or issues may arise.

\subsection{Sequence of Component/Function Integration}
\subsubsection{Software Integration Sequence}
\subsubsection{Subsystem Integration Sequence}
