\section{Introduction}
  \subsection{Revision History}
  \subsection{Purpose and Scope}
    \subsubsection{Purpose}
      The purpose of this document is to guide how to accomplish the integration test of the myTaxiService software.
      This document will show how our software should respond to given inputs, and all the software and strategy needed to test the software.
    \subsubsection{Scope}
      myTaxiService is a tool used in the city of X in order to simplify taxi usage for customers and taxi drivers. Our tests will aim to verify the correct behaviour of 
	  each step in the booking process of a ride.
  \subsection{List of Definitions and Abbreviations}
    \begin{description}
    	\item[Guest]
    		A guest to the service, namely someone who may be a driver or a registered user but has not been authenticated.
    	\item[Registered User/User]
    		A user of the service who has gone through the registration process, whose information is stored within the service's database.
    	\item[Driver]
    		A user of the service that provides their taxi for the service. They are registered in a manner different from that of other users.
    	\item[Customer]
    		Any person trying to use a taxi through the application, both a registered users and guests.
    	\item[Ride]
    		Generic term to talk about a taxi trip
    	\item[Instant Ride]
    		A ride ordered directly through the application which instantly assigns a taxi to a customer.
    	\item[Reservation/Booked Ride]
    		A ride booked in advance through the application.
    	\item[Ride Locking/Locked Ride]
    		Term used to indicate reservations that have been "locked" from cancellation. A reservation is considered locked when it can no
    		longer be cancelled or modified (2 hours before the start of the trip).
    \end{description}
  \subsection{List of Reference Documents}
