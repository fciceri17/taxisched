\section{Introduction}

\subsection{Purpose}
	This DD is a specification document intended to outline the design choices being made to implement "myTaxiService". This includes a detailed component view, information about
	patterns used and stylistic choices, a description of the algorithms to be used and an overview of the UI. These specifications are for the developers that will be writing the code
	for the service.
\subsection{Scope}
	This service will require the development of the following:
	\begin{description}
		\item[Client-Side] \hfill
			\begin{itemize}
				\item A mobile application for customers
				\item A web interface for customers to access the service
				\item A mobile application for taxi drivers to interact with calls.
			\end{itemize}
		\item[Server-Side] \hfill
			\begin{itemize}
				\item A server application that handles live queues
				\item A database that stores user data, including credentials
				\item Another database that handles Registered user rides, memorizing both booked rides and past itineraries
			\end{itemize}
	\end{description}
	
\subsection{Definition, Acronyms and Abbreviations}
\begin{description}
	\item[Guest]
		A guest to the service, namely someone who may be a driver or a registered user but has not been authenticated.
	\item[Registered User/User]
		A user of the service who has gone through the registration process, whose information is stored within the service's database.
	\item[Driver]
		A user of the service that provides their taxi for the service. They are registered in a manner different from that of other users.
	\item[Customer]
		Any person trying to use a taxi through the application, both a registered users and guests.
	\item[Ride]
		Generic term to talk about a taxi trip
	\item[Instant Ride]
		A ride ordered directly through the application which instantly assigns a taxi to a customer.
	\item[Reservation/Booked Ride]
		A ride booked in advance through the application.
	\item[Ride Locking/Locked Ride]
		Term used to indicate reservations that have been "locked" from cancellation. A reservation is considered locked when it can no
		longer be cancelled or modified (2 hours before the start of the trip).
\end{description}

\subsection{Document Structure}
	The remainder of the document is divided into the following sections:
	\begin{enumerate}
	\item \textbf{Introduction} which contains basic information on how the service works
	\item \textbf{Architectural description} where the components of the service are described, along with the different views, the stylistic choices and the design choices taken.
	\item \textbf{Algorithm design} the section where a basic description of the algorithms being used is given.
	\item \textbf{UX} the section concerning the UI design choices accompanied by some mockups.
	\item \textbf{Requirements traceability} where requirement from the RASD are mapped to the specifications within this document.
	\end{enumerate}
