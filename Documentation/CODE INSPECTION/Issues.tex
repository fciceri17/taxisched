\section{Code Issues}
From now on in this section, we will change the way how we call the methods in order to simplify the reading and the comprehension. \newline
In particular we will refer to the method:
\begin{itemize}
	\item \textit{HandlerProcessingResult processAnnotation $$($$AnnotationInfo annInfo$$)$$} as Method 1
	\item \textit{Boolean ignoreWebserviceAnnotations $$($$AnnotatedElement annElem,AnnotatedElementHandler annCtignoreWebserviceAnnotationsx$$)$$} as Method 2
	\item \textit{Boolean isJaxwsRIDeployment $$($$AnnotationInfo annInfo$$)$$} as Method 3
\end{itemize}

\newpage
\begin{tabular}{|c|p{4cm}|p{4cm}|p{4cm}|}
\hline
 & Method 1 & Method 2 & Method 3\\
\hline
1 & ok & ok & ok\\
\hline
2 & ok & ok & ok\\
\hline
3 & ok & ok & ok\\
\hline
4 & ok & ok & ok\\
\hline
5 & ok & ok & ok\\
\hline
6 & ok & ok & ok\\
\hline
7 & ok & ok & ok\\
\hline
8 & There is an inconsistent indentation in line 281 & ok & ok \\
\hline
9 & ok & ok  & ok \\
\hline
10 & ok & ok & ok \\
\hline
11 & ok & ok & ok \\
\hline
12 & ok & ok & ok \\
\hline
13 & Line lenght exceeds 80 characters & Line lenght exceeds 80 characters & Line lenght exceeds 80 characters \\
\hline
14 & Line lenght does exceed 120 characters in multiple line (132-153-301-320) & ok & ok \\
\hline
15 & see below & ok & ok \\
\hline
16 & ok & ok & ok \\
\hline
17 & ok & ok & ok \\
\hline
18 & Comments could be more professional & Comments could be more professional & No comments \\
\hline
19 & ok & ok & ok \\
\hline
20 & ok & ok & ok \\
\hline
21 & ok & ok & ok \\
\hline
22 & Not implemented consistently & ok & ok \\
\hline
23 & Incomplete javadoc & Incomplete javadoc & Incomplete javadoc \\
\hline
24 & ok & ok & ok \\
\hline
25 & ok & ok & ok \\
\hline
26 & ok & ok & ok \\
\hline
27 & The method could be broken down in multiple methods & Dupicated method in WebServiceProviderHandler.java & Dupicated method in WebServiceProviderHandler.java \\
\hline
28 & ok & ok & ok \\
\hline
29 & ok & ok & ok \\
\hline
30 & ok & ok & ok \\
\hline
\end{tabular}
\newpage

\begin{tabular}{|c|p{4cm}|p{4cm}|p{4cm}|}
\hline
 & Method 1 & Method 2 & Method 3\\
\hline
31 & ok & ok & ok \\
\hline
32 & ok & ok & ok \\
\hline
33 & See below & ok & ok \\
\hline
34 & ok & ok & ok \\
\hline
35 & ok & ok & ok \\
\hline
36 & ok & ok & ok \\
\hline
37 & ok & ok & ok \\
\hline
38 & ok & ok & ok \\
\hline
39 & ok & ok & ok \\
\hline
40 & See below & See below & See below \\
\hline
41 & ok & ok & ok \\
\hline
42 & ok & ok & ok \\
\hline
43 & ok & ok & ok \\
\hline
44 & ok & ok & ok \\
\hline
45 & ok & ok & ok \\
\hline
46 & ok & ok & ok \\
\hline
47 & ok & ok & ok \\
\hline
48 & ok & ok & ok \\
\hline
49 & ok & ok & ok \\
\hline
50 & Used generic exception e & Used generic exception e & ok \\
\hline
51 & ok & ok & ok \\
\hline
52 & Used generic exception e & Used generic exception e & ok \\
\hline
53 & ok & No action performed & ok \\
\hline	
54 & ok & ok & ok \\
\hline
55 & ok & ok & ok \\
\hline
56 & ok & ok & ok \\
\hline
57 & ok & ok & ok \\
\hline
58 & ok & ok & ok \\
\hline
59 & ok & ok & ok \\
\hline
60 & ok & ok & ok \\
\hline
\end{tabular}

\subsection{Indentation}

 The indentation is generally well used except in line 281 of Method 1.

\subsection{File Organization}

Several issues arose analyzing the length of lines of code. In more than one instance, these not only exceeded the 80 character standard, but even the 120 character limit. Examples inlude
lines 135-136, 154, 301, 320. In these cases, the length of the methods being called was too long and fragmenting the code would look very dirty. Curiously, at line 143 there seesm to be
a typing mistake, where almost 40 empty spaces appear at the end of the line of code. \\
In other cases, like lines 218 and 297, the lines of code are longer than 80 characters but only marginally so, under the 120 character recommendation. 
\subsection{Wrapping lines}

Sometimes there are more than one line breaks or comments after a set of operations is closed,
like "if" and variable declarations/initializations. However, this is acceptable since it simplifies code reading.

\subsection{Comments}
Most comments are well placed and help clarify the code in the analyzed methods. However, in some instances the code like at line 155, some comments come across as unprofessional, with
improper usage of exclamation marks and the like. Also, the comment in line 604(method 2) does not clearly explain what the program does when an exception is caught.
\subsection{Java Source Files}

During our analysis two problems arose with the java source files. The first problem was improper implementation. While both classes implemented an abstract class, 	only one of them
properly overrides the methods. In the WebServiceProviderHandler class, method 1(line 107) and the inner class (line 101)
\begin{lstlisting}
	public Class<? extends Annotation>[] 
	getTypeDependencies()
\end{lstlisting}

are both missing the @Override flag.

Also, the javadoc is incomplete for a few private methods, method 2 and method 3. While not mandatory, it is incoherent. Firstly, it is only written for one of two methods.
Furthermore, in the case of method 3, at lines 589-591, the javadoc only semantically explains the behavior of the code. The significance of the parameter and of the return value are not
explicitly stated. The same thing happens in the WebServiceProviderHandler class, where the same methods exist and are documented in the same manner.
\\
Another problem found with the document was the length of method 1. While it is mostly just calls to other methods, in some cases it is a long list checks being done within the code. Each
one of these if-statement chains could have been moved to a separate method, for clarity and an easier comprehension of the code.
\subsection{Class and Interface Declarations}

Classes are correctly done and there are no issues with the declarations barring the notation errors described in the previous subsection. However, there seems to be some duplicate code
in the WebServiceHandler class and the WebServiceProviderHandler class, where methods are duplicated. While method 1 is different in each file, methods 2 and 3 are exactly the same, and
as a result should be placed in another class that may be accessed by both.
Furthermore the order in which methods 2 and 3 are within the 2 analyzed classes is inverted, which seems to be an incoherent choice.
\subsection{Initializations and declarations}

Declarations and initializations are mostly done in the right way. The only problem noticed is that in Method 1 there are a lot of declarations that aren't at the beginning, 
but could be a right choice in order to make the code more readeable; being the method very long, create all the variables hundrends of lines before using them could confuse.
\subsection{Object Comparison}

The comparisons are always done in the correct way. Comparsions with == are only used to check if an object is equal to \textit{null}.
\subsection{Exceptions}

Exception are used in a very generic way, most of the time using a generic catch with the Exception \textit{e} and without performing any action.\newline
Only once a catch throws the more specific Exception \textit{AnnotationProcessorException} in Method 1, at line 169.
