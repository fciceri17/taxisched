\section{Assigned Classes and Methods}

Web Service Annotation: Both classes assigned to our group were related to javax.jws.WebService annotations.
The annotation process and the WebService annotation style are described briefly below
\subsection{Annotation}
	Annotations are files containing metadata about blocks of code or entire files. More information about annotations can be found on 
	the official java site in the tutorial section at \url{https://docs.oracle.com/javase/tutorial/java/annotations/}.
\newpage
\subsection{javax.jws.WebService}
	The WebService annotation type is used when referring to Java classes and interfaced that implement a Web Service or a Web Service interface.
	The WebService annotation is built as follows: \\
	
	\tiny{}
	\begin{tabular}{|p{2cm}|p{11cm}|}
	\hline
	Annotation Name & Description\\
	\hline
	name 			&	The name of the Web Service. \\
	\hfill			&  \\ \hfill
	\hfill			&	Used as the name of the wsdl:portType when mapped to WSDL 1.1. \\
	\hline
	targetNamespace & 	If the @WebService.targetNamespace annotation is on a service endpoint interface, the targetNamespace is used for the namespace
						for the wsdl:portType (and associated XML elements). \\
	\hfill			&  \\ \hfill
	\hfill			&	If the @WebService.targetNamespace annotation is on a service implementation bean that does NOT reference a service endpoint interface 
						(through the endpointInterface attribute), the targetNamespace is used for both the wsdl:portType and the wsdl:service (and associated XML elements). \\
	\hfill			&  \\ \hfill
	\hfill			&	If the @WebService.targetNamespace annotation is on a service implementation bean that does reference a service endpoint interface 
						(through the endpointInterface attribute), the targetNamespace is used for only the wsdl:service (and associated XML elements). 
						\\
	\hline
	serviceName & 	The service name of the Web Service. \\
	\hfill			&  \\ \hfill
	\hfill			&	Used as the name of the wsdl:service when mapped to WSDL 1.1.\\
	\hline
	portName &  	The port name of the Web Service. \\
	\hfill			&  \\ \hfill
	\hfill			&	Used as the name of the wsdl:port when mapped to WSDL 1.1. \\
	\hline
	wsdlLocation & 	The location of a pre-defined WSDL describing the service. \\
	\hfill			&  \\ \hfill
	\hfill			&	The wsdlLocation is a URL (relative or absolute) that refers to a pre-existing WSDL file. The presence of a wsdlLocation value indicates that
					the service implementation bean is implementing a pre-defined WSDL contract. The JSR-181 tool MUST provide feedback if the service implementation 
					bean is inconsistent with the portType and bindings declared in this WSDL. Note that a single WSDL file might contain multiple portTypes and multiple bindings.
					The annotations on the service implementation bean determine the specific portType and bindings that correspond to the Web Service. \\
	\hline
	endpointInterface 	& The complete name of the service endpoint interface defining the service?s abstract Web Service contract. \\
	\hfill				&  \\ \hfill
	\hfill				& 	This annotation allows the developer to separate the interface contract from the implementation. If this annotation is present,
						the service endpoint interface is used to determine the abstract WSDL contract (portType and bindings). The service endpoint interface
						MAY include JSR-181 annotations to customize the mapping from Java to WSDL. \\
	\hfill				&  \\ \hfill
	\hfill				&	The service implementation bean MAY implement the service endpoint interface, but is not REQUIRED to do so.\\
	\hfill				&  \\ \hfill
	\hfill				&	If this member-value is not present, the Web Service contract is generated from annotations on the service implementation bean. If a service
						endpoint interface is required by the target environment, it will be generated into an implementation-defined package with an implementation-
						defined name. \\
	\hline
	\end {tabular}
	\normalsize
	\\
	All this information may be found on the Java EE site at \url{https://docs.oracle.com/javaee/5/api/javax/jws/WebService.html}.