	\section{Introduction}

\subsection{Purpose}
	This RASD is a specification document intended to outline the measures being taken to implement "myTaxiService", a service which will be used
	by taxis in the city of X in order to facilitate taxi usage for drivers and customers. These specifications are for the government of X, the
	future provider of this service, and the developers that will create the applications to support the service.
%\endsubsection   
\subsection{Scope}
	\subsubsection{Software}
		This service will require the development of the following:
		\begin{description}
			\item[Client-Side] \hfill
				\begin{itemize}
					\item A mobile application for customers
					\item A web interface for customers and drivers to access the service
					\item A mobile application for taxi drivers to interact with calls.
				\end{itemize}
			\item[Server-Side] \hfill
				\begin{itemize}
					\item A server application that handles live queues
					\item A database that stores user data, including credentials
					\item Another database that handles Registered user rides, memorizing both booked rides and past itineraries
				\end{itemize}
		\end{description}
		The service will allow customers to instantly call a taxi, without requiring authentication. In case of advance booking, users need to login
		first, and can then access authenticated user functions. These include for example being able to book a taxi for a future date.\\
		When booking taxis, a registered user may enable ride-sharing. Ride-sharing queues up a customer's ride into the system, which will handle merging multiple requests into
		a single unified ride, which shall only marginally alter each single request by a small amount, determined by some predefined parameters.\\
		Authenticated users can also modify their already booked rides, as well as cancel them within a time frame of 2 hours from the start of their ride, and also modify
		their personal and payment information.\\
		Taxi drivers will have an interface which allows them to communicate when they start and stop working(being available), and the menu to accept or refuse a 
		job. Drivers will also have an option to remove themselves from the queue, for the occasions when they pick up passengers from the street.
		The system is responsible for sending the appropriate notifications to the customer when their ride has been accepted, along with information
		on the taxi coming to pick them up,	as well as provide an estimate of the time remaining before the start of their ride.
	%\endsubsubsection
	\subsubsection{Applications}
		This service is intended to simplify the process of acquiring a taxi in X. It should remove the need of a human operator
	%\endsubsubsection
	\subsubsection{Goals}
		The service should allow users to do the following things:
		\begin{enumerate}
		\item Allow a customer to get a taxi instantly, much like dialing the taxi number in any city
		\item Allow a customer, after registration, to book a taxi for a specific date and time
		\item Allow registered users to automatically share rides with each other, through an automatic queueing system
		\item Allow taxi drivers to accept rides both through the application and externally in a seamless manner
		\item Assign taxi drivers to customers automatically 
		\end{enumerate}
			
	%\endsubsubsection
%\endsubsection   
\subsection{Definitions}
	\begin{description}
		\item[User]
			A guest to the service, namely someone who may be a driver or a registered user but has not been authenticated.
		\item[Registered User]
			A user of the service who has gone through the registration process, whose information is stored within the service's database.
		\item[Driver]
			A user of the service that provides their taxi for the service. They are registered in a manner different from that of other users.
		\item[Customer]
			Any person trying to use a taxi through the application, both a registered users and guests.
	\end{description}
%\endsubsection
\subsection{Overview}
	This document is followed by a few more sections.
	The next section describes the service's structure, how the service operates, what each type of user can do and the way the
	back-end infrastructure operates. \\
	The following one analyses what is going to be required to make the actual application work.
	The final sections are composed of  use case and sequence diagrams, and an alloy analysis, intended to describe the application's behaviour upon user
	interaction and analyse the consistency of the service's intended behaviour.
%\endsubsection

